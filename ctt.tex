\documentclass[main.tex]{subfiles}

\begin{document}
\onehalfspacing

\chapter{Computational Type Theories}

As alluded to at the end of the previous chapter, we may add a judgement
\ver{M}{A} which deals directly with the objects $M$ which verify the
propositions (or types) $P$. We will develop a full theory of dependent types
in the sense of Martin-L\"of.\footnote{Please note that the judgements given here, and
their meaning explanations, are \emph{not} the same as those used in Constable
et al's ``Computational Type Theory'' and Nuprl. In this chapter, we use the
term ``computational type theory'' in a general sense to characterize a family
of type theories which have their origin in Martin-L\"of's 1979 paper
\emph{Constructive Mathematics and Computer Programming}.}

\section{The categorical judgements}

Because we will need to consider the introduction of types which do not have a
trivial (intensional) equality relation, we must first amend the meaning
explanations for some of our judgements, and add a few new forms of judgement.
First, we will refer to \emph{types} rather than \emph{propositions} in order
to emphasize the generality of the theory; in some presentations, the word
\emph{set} is used instead.

The meaning of hypothetical and general judgement are the same as in
the previous chapter, and so we will not reproduce them here. The
first form of judgement is \framebox{\isset{A}}, and its meaning
explanation is as follows:

\begin{quote}
  To know $\isset{A}$ is to know an $A'$ such that $\reduce{A}{A'}$,
  and you know what counts as a canonical verification of $A'$ and when
  two such verifications are equal.
\end{quote}

The next form of judgement is \framebox{\ver{M}{A}}, which remains
the same as before:

\begin{quote}
  To know $\ver{M}{A}$ (presupposing $\isset{A}$) is to know an $M'$ such that
  $\reduce{M}{M'}$ and $M'$ is a canonical verification of $A$.
\end{quote}

We'll need to add new judgements for equality (equality of
verifications, and equality of types respectively). First, equality of
verifications is written \framebox{$\ever{M}{N}{A}$}, and means the
following:

\begin{quote}
  To know $\ever{M}{N}{A}$ (presupposing $\ver{M}{A}$ and $\ver{N}{A}$,
  and thence $\isset{A}$) is to know that the values of $M$ and $N$ are
  equal as canonical verifications of $A$.
\end{quote}

The fact that $M$ and $N$ reduce to canonical values which verify $A$
is known from the presuppositions of the judgement; and what it means
for them to be equal as such is known from the evidence of the
presupposition $\isset{A}$ which is obtained from the other
presuppositions.

For equality of types \framebox{$\eisset{A}{B}$}, there are a number of
possible meaning explanations, but we'll use the one that Martin-L\"of
used starting in 1979:

\begin{quote}
  To know $\eisset{A}{B}$ (presupposing $\isset{A}$ and $\isset{B}$) is to know
$\genj{M}{\hyp{\ver{M}{B}}{\ver{M}{A}}}$ and
$\genj{M}{\hyp{\ver{M}{A}}{\ver{M}{B}}}$, and moreover
$\genj{M,N}{\hyp{\ever{M}{N}{B}}{\ever{M}{N}{A}}}$ and
$\genj{M,N}{\hyp{\ever{M}{N}{A}}{\ever{M}{N}{A}}}$.
\end{quote}

In other words, two types are equal when they have the same canonical
verifications, and moreover, the same equality relation over their canonical
verifications. Note that there are other possible explanations for type
equality, including more intensional ones that appeal to the syntactic
structure of type expressions, and these turn out to be more useful in proof
assistants for practical reasons. However, the extensional equality that we
have expounded is the easiest and most obvious one to formulate, so we will
use it here.

\section{The functional sequent judgement}

Now, because we are allowing the definition of types with arbitrary equivalence
relations, we cannot use plain hypothetico-general judgement in the course of
defining our types. For instance, if we were going to try and define the
function type $A\supset B$ in the same way as we did in the previous chapter,
we would permit ``functions'' which are not in fact functional, i.e.\ they do
not take equal inputs to equal outputs. As such, we will need to bake
functionality (also called extensionality) into the definition of functions,
and since we will need this in many other places, we elect to simplify our
definitions by baking it into a single judgement which is meant to be used
instead of plain hypothetico-general judgement.

The judgement which expresses simultaneously generality, hypothesis and
functionality has been written in multiple ways. Martin-L\"of has always
written it as $\hyp{\mathcal{J}}{\Gamma}$, but this is a confusing notation
because it appears as though it is merely a hypothetical judgement (but it is
much more, as will be seen). Very frequently, it is written with a turnstile,
$\Gamma\vdash\mathcal{J}$, and in the early literature surrounding Constable's
Computational Type Theory and Nuprl, it was written
\framebox{$\sequent{\Gamma}{\mathcal{J}}$}; we choose this last option to avoid
confusion with a similar judgement form which appears in proof-theoretic,
intensional type theories; we'll call the judgement form a ``(functional) sequent''.

First, we will add the syntax of contexts into the computation system:
\begin{gather*}
  \reduce{\cdot}{\cdot}\tag{Canonical}\\
  \reduce{(\Gamma,x:A)}{(\Gamma,x:A)}
\end{gather*}

The sequent judgements will be defined simultaneously with two other
judgements, \framebox{$\isctx{\Gamma}$} (``$\Gamma$ is a context'')
and \framebox{$\fresh{x}{\Gamma}$} (``$x$ is fresh in
$\Gamma$''). Their meaning explanations are as follows:

\begin{quote}
  To know $\isctx\Gamma$\ is to know that $\reduce{\Gamma}{\cdot}$, or it is to know
  a variable $x$ and expressions $\Delta,A$ such that $\reduce{\Gamma}{\Delta, x:A}$
  and $\sequent{\Delta}{\isset{A}}$, and $\fresh{x}{\Delta}$.
\end{quote}

\begin{quote}
  To know $\fresh{x}{\Gamma}$\ (presupposing $\isctx\Gamma$) is to know that
  $\reduce{\Gamma}{\cdot}$, or, if $\reduce{\Gamma}{\Delta,y:A}$
  such that $x$ is not $y$ and $\fresh{x}{\Delta}$.
\end{quote}

In other words, the well-formed contexts are inductively generated by the
following grammar:
\begin{gather*}
  \infer{\isctx\cdot}{}
  \qquad
  \infer{
    \isctx{\Gamma,x:A}
  }{
    \isctx\Gamma &
    \sequent{\Gamma}{\isset{A}} &
    \fresh{x}{\Gamma}
  }\\
  \infer{\fresh{x}{\cdot}}{}
  \qquad
  \infer{
    \fresh{x}{\Gamma,y:A}
  }{
    \fresh{x}{\Gamma}
  }
\end{gather*}

We will say that $\sequent{\Gamma}{\mathcal{J}}$ is only a judgement under the
presuppositions that $\isctx\Gamma$\ and that $\mathcal{J}$ is a categorical
judgement of the form $\isset{A}$, $\eisset{A}{B}$, $\ver{M}{A}$, or
$\ever{M}{N}{A}$. Because of this presupposition, we are justified in
propounding the meaning of the judgement by structural induction on the
evidence for $\isctx\Gamma$, separately for each $\mathcal{J}$.

Now we may begin giving the meaning explanations for
$\sequent{\Gamma}{\mathcal{J}}$, starting with typehood with respect to a context:

\begin{quote}

  To know $\sequent{\Gamma}{\isset{A}}$ (presupposing $\isctx\Gamma$) is to
know, if $\reduce{\Gamma}{\cdot}$, then $\isset{A}$; otherwise, if
$\reduce{\Gamma}{\Delta,x:B}$, that
$\genj{M}{\hyp{\sequent{\Delta}{[M/x]\isset{A}}}{\sequent{\Delta}{M:B}}}$ and
moreover
$\genj{M,N}{\hyp{\sequent{\Delta}{\eisset{[M/x]A}{[N/x]A}}}{\sequent{\Delta}{\ever{M}{N}{B}}}}$.

\end{quote}

In this way, we have defined the sequent judgement by structural induction on
the evidence for the presupposition $\isctx\Gamma$. We can explain type
equality sequents in a similar way:

\begin{quote}
  To know $\sequent{\Gamma}{\eisset{A}{B}}$ (presupposing $\isctx\Gamma$, $\sequent{\Gamma}{\isset{A}}$, $\sequent{\Gamma}{\isset{B}}$) is to know, if $\reduce{\Gamma}{\cdot}$, that $\eisset{A}{B}$; otherwise, if $\reduce{\Gamma}{\Delta,x:C}$, it is to know
  \[
    \genj{M}{
      \hyp{
        \sequent{\Delta}{
          \eisset{[M/x]A}{[M/x]B}
        }
      }{
        \sequent{\Delta}{
          \ver{M}{C}
        }
      }
    }
  \]
  and moreover, to know
  \[
    \genj{M,N}{
      \hyp{
        \sequent{\Delta}{
          \eisset{[M/x]A}{[N/x]B}
        }
      }{
        \sequent{\Delta}{
          \ever{M}{N}{C}
        }
      }
    }
  \]
\end{quote}

Next, the meaning of membership sequents is explained:

\begin{quote}
  To know $\sequent{\Gamma}{\ver{M}{A}}$ (presupposing $\isctx\Gamma$, $\sequent{\Gamma}{\isset{A}}$) is to know, if
  $\reduce\Gamma\cdot$, that $\ver{M}{A}$; otherwise, if
  $\reduce{\Gamma}{\Delta,x:B}$, it is to know
  \[
    \genj{N}{
      \hyp{
        \sequent{\Delta}{
          \ver{[N/x]M}{[N/x]A}
        }
      }{
        \sequent{\Delta}{
          \ver{N}{B}
        }
      }
    }
  \]
  and moreover, to know
  \[
    \genj{L,N}{
      \hyp{
        \sequent{\Delta}{
          \ever{[L/x]M}{[N/x]M}{A}
        }
      }{
        \sequent{\Delta}{
          \ever{L}{N}{B}
        }
      }
    }
  \]
\end{quote}

Finally, member equality sequents have an analogous explanation:

\begin{quote}
  To know $\sequent{\Gamma}{\ever{M}{M'}{A}}$ (presupposing $\isctx\Gamma$, $\sequent{\Gamma}{\isset{A}}$) is to know, if
  $\reduce{\Gamma}{\cdot}$, that $\ever{M}{M'}{A}$; otherwise, if
  $\reduce{\Gamma}{\Delta,x:B}$, it is to know
  \[
    \genj{N}{
      \hyp{
        \sequent{\Delta}{
          \ever{[N/x]M}{[N/x]M'}{A}
        }
      }{
        \sequent{\Delta}{
          \ver{N}{B}
        }
      }
    }
  \]
  and moreover, to know
  \[
    \genj{L,N}{
      \hyp{
        \sequent{\Delta}{
          \ever{[L/x]M}{[N/x]M'}{A}
        }
      }{
        \sequent{\Delta}{
          \ever{L}{N}{B}
        }
      }
    }
  \]
\end{quote}

The simultaneous definition of multiple judgements may seem at first
concerning, but it can be shown to be non-circular by induction on the length
of the context $\Gamma$.

\section{The definitions of types}

We will now define the types of a simple computational type theory without
universes. In the course of doing so, opportunities will arise for further
clarifying the position of the judgements, meaning explanations and proofs on
the one hand, and the propositions, definitions and verifications on the other
hand.

\subsection{The unit type}

First, we introduce two canonical forms with trivial reduction rules:
\begin{equation}
  \reduce\tyunit\tyunit\qquad
  \reduce\bullet\bullet
  \tag{Canonical}
\end{equation}

Next, we intend to make the judgement $\isset\tyunit$\ evident; and this is done by
defining what counts as a canonical verification of $\tyunit$\ and when two such
verifications are equal. To this end, we say that $\bullet$ is a canonical
verification of $\tyunit$, and that it is equal to itself.  I wish to emphasize
that this is the entire definition of the type: we have introduced syntax, and
we have defined the canonical forms, and there is nothing more to be done.

In the presentations of type theory which are currently in vogue, a type is
``defined'' by writing out a bunch of inference rules, but in type theory, the
definitions that we have given above are prior to the rules, which are
justified in respect of the definitions and the meaning explanations of the
judgements. For instance, based on the meaning of the various forms of sequent
judgement, the following rule schemes are justified:
\begin{gather*}
  \infer{
    \sequent\Gamma\isset\tyunit
  }{
  }\qquad
  \infer{
    \sequent\Gamma\eisset{\tyunit}{\tyunit}
  }{
  }\\[6pt]
  \infer{
    \sequent\Gamma\ver\bullet\tyunit
  }{
  }\qquad
  \infer{
    \sequent\Gamma\ever{\bullet}{\bullet}{\tyunit}
  }{
  }
\end{gather*}

Each of the assertions above has evidence of a certain kind; since the
justification of these rules with respect to the definitions of the logical
constants and the meaning explanations of the judgements is largely
self-evident, we omit it in nearly all cases. It is just important to remember
that it is not the rules which define the types; a type $A$ is defined in the
course of causing the judgement $\isset{A}$ to become evident.  These rules
merely codify standard patterns of use, nothing more, and they must each be
justified.

\subsection{The empty type}

The empty type is similarly easy to define. First, we introduce a constant:
\begin{gather*}
  \reduce\tyvoid\tyvoid\tag{Canonical}
\end{gather*}

To make the judgement $\isset\tyvoid$\ evident, we will say that there are no
canonical verifications of \tyvoid, and be done with it. This definition
validates some further rules schemes:
\begin{gather*}
  \infer{
    \sequent\Gamma\isset\tyvoid
  }{
  }\qquad
  \infer{
    \sequent\Gamma\eisset{\tyvoid}{\tyvoid}
  }{
  }\\[6pt]
  \infer{
    \sequent\Gamma\mathcal{J}
  }{
    \sequent\Gamma\ver{M}{\tyvoid}
  }
\end{gather*}

The last rule simply says that if we have a verification of $\tyvoid$, then we
may conclude any judgement whatsoever. Remember that the inference rules are
just notation for an \emph{evident} hypothetical judgement, e.g.\
$\hyp{\sequent{\Gamma}\mathcal{J}}{\sequent\Gamma\ver{M}{\tyvoid}}$.

Note that we did not introduce any special constant into the
computation system to represent the elimination of a verification of
$\tyvoid$\ (in proof-theoretic type theories, this non-canonical form
is usually called $\mathsf{abort}(R)$).  This is because,
computationally speaking, there is never any chance that we should
ever have use for such a term, since we need only consider the
evaluation of closed terms (which is guaranteed by the meaning
explanations), and by its very definition, there can never be a closed
verification of \tyvoid.

\subsection{The cartesian product of a family of types}

This will be our first dependent type, and it will likewise be the first
example of a non-trivial addition to the computation system. First, let us add
our canonical and non-canonical forms and their reduction rules:
\begin{gather*}
  \reduce{\typrod{A}{x}{B}}{\typrod{A}{x}{B}}
  \qquad
  \reduce{\lam{x}{E}}{\lam{x}{E}}
  \tag{Canonical}\\
  \infer{
    \reduce{\ap{M}{N}}{M'}
  }{
    \reduce{M}{\lam{x}{E}} &
    \reduce{[N/x]E}{M'}
  }\tag{Non-canonical}
\end{gather*}

This is the first time in this chapter that we have introduced a term former
with binding structure; it should be noted that the intensional equality of
expressions is up to alpha equivalence, and we will not pay attention to issues
of variable renaming in our presentation.

We will make evident the following judgement scheme:
\[
  \infer{
    \isset{\typrod{A}{x}{B}}
  }{
    \isset{A} &
    \sequent{x:A}{\isset{B}}
  }
\]

Or, written as a hypothetical judgement:
\[
  \hyp{\isset{\typrod{A}{x}{B}}}{\isset{A},\;\sequent{x:A}{\isset{B}}}
\]

This is to say, under the stated assumptions, we know what counts as a
canonical verification of $\typrod{A}{x}{B}$ and when two such
verifications are equal. We will say that $\lam{x}{E}$ is a canonical
verification of $\typrod{A}{x}{B}$ just when we know
$\sequent{x:A}{\ver{E}{B}}$; moreover, that two verifications
$\lam{x}{E}$ and $\lam{y}{E'}$ are equal just when
$\sequent{\ver{z}{A}}{\ever{[z/x]E}{[z/y]E'}{[z/x]B}}$.

By the meaning of the sequent judgement, this is to say that a lambda
expression must be functional with respect to its domain (i.e. it must take
equals to equals). We did not need to hypothesize directly two elements of the
domain and their equality because this is part of the meaning explanation for
the sequent judgement already. Likewise, two lambda expressions are equal when
equal inputs yield equal results in both.

The familiar inference rules, which codify the standard mode of use for the
family cartesian product, are justified by this definition:
\begin{gather*}
  \infer{
    \sequent\Gamma\isset{\typrod{A}{x}{B}}
  }{
    \sequent\Gamma\isset{A} &
    \sequent{\Gamma,x:A}{\isset{B}}
  }\qquad
  \infer{
    \sequent\Gamma\eisset{\typrod{A}{x}{B}}{\typrod{A'}{y}{B'}}
  }{
    \sequent\Gamma\eisset{A}{A'} &
    \sequent{\Gamma,z:A}{\eisset{[z/x]B}{[z/y]B'}}
  }\\[6pt]
  \infer{
    \sequent\Gamma\ver{\lam{x}{E}}{\typrod{A}{x}{B}}
  }{
    \sequent{\Gamma,x:A}{\ver{E}{B}}
  }\qquad
  \infer{
    \sequent\Gamma\ever{\lam{x}{E}}{\lam{y}{E'}}{\typrod{A}{x}{B}}
  }{
    \sequent{\Gamma,z:A}{\ever{[z/x]E}{[z/y]E'}{[z/x]B}}
  }\\[6pt]
  \infer{
    \sequent\Gamma\ver{\ap{M}{N}}{[N/x]B}
  }{
    \sequent\Gamma\ver{M}{\typrod{A}{x}{B}} &
    \sequent\Gamma\ver{N}{A}
  }\qquad
  \infer{
    \sequent\Gamma\ever{\ap{M}{N}}{\ap{M'}{N'}}{[N/x]B}
  }{
    \sequent\Gamma\ever{M}{M'}{\typrod{A}{x}{B}} &
    \sequent\Gamma\ever{N}{N'}{A'}
  }
\end{gather*}

Note that the type equality rule scheme that we gave above is structural; it is
validated by the meaning explanations, but it is by no means the full totality
of possible equalities between family cartesian product types, which is
extensional in this theory.

It will be instructive to explicitly justify the application rule above with
respect to the meaning explanations, since I have claimed that such rules are
posterior to the definitions we expounded prior to giving these rules.

\begin{proof}
It suffices to consider the case that $\reduce{\Gamma}{\cdot}$,
because hypotheses may always be added to the context (this is called
weakening). And so, by the meaning of the sequent judgement at the
empty context, the rule amounts to the assertion
\[
  \hyp{
    \ver{\ap{M}{N}}{[N/x]B}
  }{
    \ver{M}{\typrod{A}{x}{B}},\; \ver{N}{A}
  }
\]

By the meaning explanation for hypothetical judgement, and the
definition of the family cartesian product type, we know that
$\reduce{M}{\lam{x}{E}}$ for some $E$ such that we know
$\sequent{x:A}{\ver{E}{B}}$; from the meaning of the sequent
judgement, we can conclude $\genj{L}{\hyp{\ver{[L/x]E}{[L/x]B}}{L\in
A}}$. On the other hand, from the computation rules, we know that if
for some particular $E'$, $\reduce{[N/x]E}{E'}$, then we know
$\reduce{\ap{\lam{x}{E}}{N}}{E'}$; to demonstrate the evidence of the
premise, we may instantiate $L$ at $N$ to know $\ver{[N/x]E}{[N/x]B}$,
whence by the meaning of membership, we know that there exists some
canonical $E'$ such that $\reduce{[N/x]}{E'}$.
\end{proof}


\end{document}

