\documentclass{amsart}
\usepackage{amsmath}
\usepackage{amssymb}
\usepackage{dsfont}
\usepackage{multicol}
\usepackage{proof}
\usepackage{setspace}
\usepackage{stmaryrd}

\newtheorem{thm}{Theorem}[section]
\newtheorem{prop}[thm]{Proposition}
\newtheorem{lem}[thm]{Lemma}
\newtheorem{cor}[thm]{Corollary}
\theoremstyle{definition}
\newtheorem{definition}[thm]{Definition}
\newtheorem{example}[thm]{Example}

\theoremstyle{remark}
\newtheorem{remark}[thm]{Remark}
\numberwithin{equation}{section}

\newcommand\isprop[1]{\ensuremath{#1\;\mathit{prop}}}
\newcommand\istrue[1]{\ensuremath{#1\;\mathit{true}}}
\newcommand\hyp[2]{\ensuremath{#1\ [#2]}}
\newcommand\ver[2]{\ensuremath{#1\in#2}}
\newcommand\reduce[2]{\ensuremath{#1\Rightarrow#2}}

\begin{document}
\title{Type theory and its meaning explanations}
\author{Jonathan Sterling}
% \address{AlephCloud Systems}
% \email{jon@jonmsterling.com}

\maketitle

\onehalfspacing

To start, we will consider the notion of a \emph{logical theory}; in my mind,
it starts with a species (or set) of judgements that can be proposed, asserted,
and (if they are evident) known.

\section{Judgements of a Logical Theory}

The basic forms of judgement for a logical theory will be \framebox{\isprop{P}} and
\framebox{\istrue{P}}; and what is $P$? It is a member of the species of terms,
which are made meaningful in the course of making the judgement
$\isprop{P}$ evident for a proposition $P$.

To each judgement is assigned a \emph{meaning explanation}, which explicates
the knowledge-theoretic content of the judgement. For a judgement
$\mathcal{J}$, a meaning explanation should be in the form:
\begin{quote}
  To know $\mathcal{J}$ is to know...
\end{quote}

The meaning of the judgement \isprop{P} is, then, as follows:
\begin{quote}
  To know \isprop{P} is to know that $P$ is a proposition, which is to know
  what counts as a direct verification of $P$.
\end{quote}

So if a symbol $P$ is taken to denote a proposition, we must know \emph{what
sort of thing} is to be taken as a direct verification of $P$, and this is by
definition. A ``direct verification'' is understood in constrast with an
``indirect verification'', which is to be thought of as a means or plan for
verifying the proposition. Now, the judgement \istrue{P} is only meaningful in
case we know \isprop{P} (this is called a presupposition). Then the meaning of
\istrue{P} is as follows:
\begin{quote}
  To know \istrue{P} is to have a verification of $P$.
\end{quote}

From the (implicit) presupposition \isprop{P}, we already know what counts as a
verification, so the meaning explanation is well-defined. Note that it is the
same to have a means or plan for verifying $P$ as to have a (direct)
verification; this follows from the fact that one may put into action a plan
for verifying $P$ and get out such a verification, and likewise, it is possible
to propound a plan of verification by appealing to an existing verification.

The judgements we have described so far are ``categorical'' in the sense that
they are made without assumption. We will need to define a further form of
judgement, which is called ``hypothetical'', and this is the judgement under
hypothesis \framebox{\hyp{\mathcal{J}}{\mathcal{J}_1\dots\mathcal{J}_n}}. And
its meaning explanation is as follows:

\begin{quote}
  To know the judgement \hyp{\mathcal{J}}{\mathcal{J}_1\dots\mathcal{J}_n} is to know the
  categorical judgement $\mathcal{J}$ when you know the judgements
  $\mathcal{J}_1\dots\mathcal{J}_n$.
\end{quote}


\section{Propositions}
Now that we have propounded and explained the minimal system of judgements for
a logical theory, let us populate it with propositions. First, we have falsity
$\bot$, and we wish to make \isprop{\bot} evident; to do this, we simply state
what counts as a direct verification of $\bot$, which is that there can be no
direct verification of $\bot$.

The next basic proposition is trivially true $\top$, and to make \isprop{\top}
evident, we state that a direct verification of $\top$ is trivial. The meaning
of $\top$ thus validates the judgement \istrue{\top}.

Next, let us define conjunction; in doing so, we will make evident the
hypothetical judgement \hyp{\isprop{P\land Q}}{\isprop{P},\isprop{Q}};
equivalently, we can display this as a rule of inference:
\[
  \infer{
    \isprop{P\land Q}
  }{
    \isprop{P} &
    \isprop{Q}
  }
\]

A direct verification of $P\land Q$ consists in a verification of $P$
and a verification of $Q$; this validates the assertion of the judgement
\hyp{\istrue{P\land Q}}{\istrue{P},\istrue{Q}}. Because it is a valid
inference, we can write it as an inference rule:
\[
  \infer{
    \istrue{P\land Q}
  }{
    \istrue{P} &
    \istrue{Q}
  }
\]

A direct verification of $P\lor Q$ may be got either from a verification
of $P$ or one of $Q$. From this definition we know \hyp{\isprop{P\lor
Q}}{\isprop{P},\isprop{Q}}, or
\[
  \infer{
    \isprop{P\lor Q}
  }{
    \isprop{P} &
    \isprop{Q}
  }
\]
The verification conditions of disjunction give
rise to two evident judgements \hyp{\istrue{P\lor Q}}{\istrue{P}} and
\hyp{\istrue{P\lor Q}}{\istrue{Q}}, which we can write as inference rules:
\begin{gather*}
  \infer{
    \istrue{P\lor Q}
  }{
    \istrue{P}
  }
  \qquad
  \infer{
    \istrue{P\lor Q}
  }{
    \istrue{Q}
  }
\end{gather*}

Finally, we must define the circumstances under which $P\supset Q$ is a
proposition (i.e.\ when \isprop{P\supset Q} is evident). And we intend this to
be under the circumstances that $P$ is a proposition, and also that $Q$ is a
proposition assuming that $P$ is true. In other words, \hyp{\isprop{P\supset
Q}}{\isprop{P}, \hyp{\isprop{Q}}{\istrue{P}}}, or
\[
  \infer{
    \isprop{P\supset Q}
  }{
    \isprop{P} &
    \hyp{\isprop{Q}}{\istrue{P}}
  }
\]

Now, to validate this judgement will be a bit more complicated than the
previous ones. But by unfolding the meaning explanations for hypothetical
judgement, proposition-hood and truth of a proposition, we arrive at the
following explanation:
\begin{quote}
  To know \hyp{\isprop{P\supset Q}}{\isprop{P},\hyp{\isprop{Q}}{\istrue{P}}} is
  to know what counts as a direct verification of $P\supset Q$ when one knows what
  counts as a direct verification of $P$, and, when one has such a verification, what
  counts as a direct verification of $Q$.
\end{quote}

(Note that unless \istrue{P}, it need not be evident that \isprop{Q}.) Now, if
this judgement is going to be made evident, then we must indeed come up with
what should count as a direct verification of $P\supset Q$ under the
assumptions described above.

And so to have a direct verification of $P\supset Q$ is to have a
verification of $Q$ assuming that one has one of $P$; this is the meaning of
implication, and it validates the judgement \hyp{\istrue{P\supset
Q}}{\hyp{\istrue{Q}}{\istrue{P}}} (elliding the hypothesis for
propositionhood), and may be written as an inference rule as follows:
\[
  \infer{
    \istrue{P\supset Q}
  }{
    \hyp{\istrue{Q}}{\istrue{P}}
  }
\]

\section{Judgements for Verifications}

So far, we have given judgements which circumscribe what it means to be a
proposition, and thence for each proposition, we have by definition a notion of
what should count as a verification of that proposition. And by definition, to
assert the judgement \istrue{P} is to assert that one has a verification of $P$,
but we have not considered any judgements which actually describe such
verifications.

It is a hallmark of Martin-L\"of's program to resolve the contradiction between
syntax and semantics not by choosing symbols over meanings or meanings over
symbols, but by endowing symbols with meaning in the course of making evident
the true judgements. As such $P$ is a symbol, but when we assert \isprop{P} we
are saying that we know what proposition $P$ denotes.

A similar thing can be done with verifications themselves, by placing them in
the syntactic domain together with the propositional symbols. And then, we can
consider a judgement such as ``$M$ is a verification of $P$'', and in knowing
that judgement, we know what verification $M$ is meant to denote. In practice,
this judgement has been written in several ways:\\\medskip
\begin{tabular}{c|l}
  $M\in P$ & $M$ is an element of $P$\\
  $M\Vdash P$ & $M$ realizes $P$\\
  $P\ \llcorner\mathsf{ext}\; M\lrcorner$ & $P$ is witnessed by $M$
\end{tabular}

But they all mean the same thing, and so we will tentatively give the following
meaning explanation to this new judgement:
\begin{quote}
  * To know $\ver{M}{P}$ is to know that $M$ is a verification of $P$.
\end{quote}

But now that we have started to assign expressions to verifications, we must be
more careful about differentiating \emph{direct verifications} (which we will
call ``canonical'') from \emph{indirect verifications} (which we will call
``non-canonical''). So the domain of expressions must itself be accorded with a
notion of reduction to canonical form, and this corresponds with putting into
action a plan of verification in order to get an actual verification.
\begin{quote}
  To know \reduce{M}{M'} is to know that $M$ is an expression which reduces to
  a canonical form $M'$.
\end{quote}

Now, we can rewrite the previous meaning explanation as follows:
\begin{quote}
  To know \ver{M}{P} is to know an $M'$ such that \reduce{M}{M'} and $M'$ is a
  canonical (direct) verification of $P$.
\end{quote}

The meaning explanation for \isprop{P} must be accordingly modified to take
into account the computational behavior of the expression domain:
\begin{quote}
  To know \isprop{P} is to know a $P'$ such that \reduce{P}{P'} and $P'$ is a
  canonical proposition, which is to say, that one knows what counts as a
  canonical verification for $P'$.
\end{quote}

As an example, then, we will update the evidence of the assertion
\hyp{\isprop{P\supset Q}}{\isprop{P},\hyp{\isprop{Q}}{\isprop{P}}}. The meaning
of this, expanded into spoken language, is as follows:
\begin{quote}
  To know \hyp{\isprop{P\supset Q}}{\isprop{P},\hyp{\isprop{Q}}{\isprop{P}}} is
  to know what counts as a canonical verification of $P\subset Q$ under
  the circumstances that \reduce{P}{P'}, such one knows what counts as a
  canonical verification $P'$, and, if one has such a verification,
  \reduce{Q}{Q'} such that one knows what counts as a canonical verification of
  $Q'$.
\end{quote}

And this is evident, since we will say that a canonical verification of
$P\subset Q$ is an expression $\lambda x. E$ such that we know
\hyp{\ver{E}{Q}}{\ver{x}{P}}. This validates the assertion \hyp{\ver{\lambda
x.E}{P\supset Q}}{\hyp{\ver{E}{Q}}{\ver{x}{P}}}, written as an inference rule:
\[
  \infer{
    \ver{\lambda x.E}{P\supset Q}
  }{
    \hyp{\ver{E}{Q}}{\ver{x}{P}}
  }
\]

By the addition of this judgement, we have graduated from a logical theory to a
type theory, in the sense of \emph{Constructive Mathematics and Computer
Programming} (Martin-L\"of, 1979). In fact, we may dispense with the original
\istrue{P} form of judgement by \emph{defining} it in terms of the new
\ver{M}{P} judgement as follows:
\[
  \infer{
    \istrue{P}
  }{
    \ver{M}{P}
  }
\]

Further forms of judgement, such as assertions of equality between propositions
and verifications, may be added, as Martin-L\"of does. However, this is not
strictly necessary as they too may be defined in terms of the $\ver{M}{P}$
judgement in the presence of an \emph{equality} proposition; this is what is
done in Constable et al's Computational Type Theory, which has only one
primitive form of judgement.

\end{document}
