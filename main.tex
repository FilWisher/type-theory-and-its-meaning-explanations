\documentclass{amsbook}
\usepackage{amsmath}
\usepackage{amssymb}
\usepackage{dsfont}
\usepackage{multicol}
\usepackage{proof}
\usepackage{setspace}
\usepackage[usenames,dvipsnames,svgnames,table]{xcolor}
\usepackage{stmaryrd}
\usepackage{subfiles}

\newtheorem{thm}{Theorem}[section]
\newtheorem{prop}[thm]{Proposition}
\newtheorem{lem}[thm]{Lemma}
\newtheorem{cor}[thm]{Corollary}
\theoremstyle{definition}
\newtheorem{definition}[thm]{Definition}
\newtheorem{example}[thm]{Example}

\theoremstyle{remark}
\newtheorem{remark}[thm]{Remark}
\numberwithin{equation}{section}

\newcommand\isprop[1]{\ensuremath{#1\;\mathit{prop}}}
\newcommand\isset[1]{\ensuremath{#1\;\mathit{type}}}
\newcommand\istrue[1]{\ensuremath{#1\;\mathit{true}}}
\newcommand\isctx[1]{\ensuremath{#1\;\mathit{ctx}}}
\newcommand\fresh[2]{\ensuremath{#1\mathrel{\#}#2}}
\newcommand\hyp[2]{\ensuremath{#1\ (#2)}}
\newcommand\genj[2]{\ensuremath{\lvert_{#1}\,#2}}
\newcommand\ver[2]{\ensuremath{#1\in#2}}
\newcommand\reduce[2]{\ensuremath{#1\Rightarrow#2}}
\newcommand\naturals{\ensuremath{\mathbb{N}}}
\newcommand\suc[1]{\ensuremath{\mathsf{S}(#1)}}
\newcommand\natrec[3]{\ensuremath{\mathsf{rec}_\naturals(#1;#2;#3)}}
\newcommand\lfhyp[2]{\ensuremath{#1\ [#2]}}
\newcommand\prf[1]{\ensuremath{\mathsf{Prf}(#1)}}
\newcommand\MLLF{\textbf{MLLF}}
\newcommand\infers[2]{\ensuremath{#1\uparrow{\color{Gray}{#2}}}}
\newcommand\checks[2]{\ensuremath{#1\downarrow#2}}

\newcommand\type[1]{\ensuremath{#1:\mathit{type}}}
\newcommand\product[2]{\ensuremath{#1\&#2}}
\newcommand\pair[2]{\ensuremath{\langle #1,#2\rangle}}
\newcommand\fst[1]{\ensuremath{\mathsf{fst}(#1)}}
\newcommand\snd[1]{\ensuremath{\mathsf{snd}(#1)}}
\newcommand\abort[2]{\ensuremath{\mathsf{abort}(#1;#2)}}

\begin{document}
\title{Type theory and its meaning explanations}
\author{Jonathan Sterling}

\thanks{Thanks to Bob Harper, Peter Dybjer, Bengt Norstr\"om and Darin Morrison
for invaluable conversations which helped sharpen my view of the meaning
explanations for computational and intensional type theories.}

% \address{AlephCloud Systems}
% \email{jon@jonmsterling.com}

\maketitle

\begin{abstract}
  At the heart of intuitionistic type theory lies an intuitive semantics called
  the ``meaning explanations''; crucially, when meaning explanations are taken as
  definitive for type theory, the core notion is no longer ``proof'' but
  ``verification''. We'll explore how type theories of this sort arise naturally
  as enrichments of logical theories with further judgements, and contrast this
  with modern proof-theoretic type theories which interpret the judgements and
  proofs of logics, not their propositions and verifications.
\end{abstract}

\onehalfspacing

\tableofcontents

\subfile{logicaltheory}
\subfile{ctt}
\subfile{prooftheory}

\end{document}
